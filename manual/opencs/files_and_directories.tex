\section{Files and Directories}
\subsection{Introduction}
This section of the manual covers usage of files and directories by the OpenCS. Files and directories are file system concepts,
and you are probably already familiar with it. We won't try to explain this concepts, we will just focus on \OCS.

\subsection{Used terms} %TODO

\subsection{Basics}

\paragraph{Directories}
OpenMW and \OCS{} uses multiple directories on file systems. First of, there is a \textbf{user directory} that holds configuration
files and few different folders. The location of the user directory is hard coded for each supported operating system.

%TODO list paths.
In addition to this single hard coded directory, both \OMW{} and \OCS{} need a~place to seek for actual data files of the game:
textures, models, sounds and files that store records of objects in game; dialogues and so one -- so called content files. We support
multiple such paths (we call it \textbf{data paths}) as specified in the configuration. Usually one data path points to the directory
where original \MW{} is either installed or unpacked. You are free to specify as many data paths as you would like,
however, there is one special data path that, as described later, is used to store newly created content files.

\paragraph{Content files}
\BS{} \MW{} engine is using two types of files: ESM (master) and ESP (plugin). The distinction between those
is not clear, and often confusing. You would expect the ESM (master) file is used to specify one master, that is modified by the ESPs plugins,
and indeed: this is the basic idea. However, original expansions also were made as ESM files, even though they essentially could be
described as a really large plugins, and therefore rather use ESP files. There were technical reasons behind this decision -- somewhat valid
in the case of original engine, but clearly it's better to create a system that can be used is more sensible way. \OMW{} achieves
this with our own content file types.

We support both ESM and ESP files, but in order to make use of new features of OpenMW one should consider using new file types designed
with our engine in mind: game files and addon files together called ``content files``.

\subparagraph{OpenMW content files}
Game and Addon files are concept somewhat similar to the old ESM/ESP, only in the way it should be from the very beginning. Nothing easier
to describe. If you want to make new game using \OMW{} as engine (so called ``total conversion'') you should create a game file.
If you want to create a addon for existing game file -- simply create addon file. Nothing else matters: The only distinction you should
consider is if your project is about changing other game, or creating a new one. Simple as that.

Other simple thing about content files are extensions. We are using .omwaddon for addon files and .omwgame for game files.

%TODO describe what content files contains. and what not.
\subparagraph{\MW{} content files}
Using our content files is recommended solution for projects that are intended to used with \OMW{} engine. However some players
wish to use original \MW{} engine, even with it large flaws and lacking features\footnote{If this is actually wrong, we are very
successful project. Yay!}. Also, since 2002 thousands of ESP/ESM files were created, some with really outstanding content.
Because of this \OCS{} simply has no other choice but support ESP/ESM files. However, if you decided to choose ESP/ESM file instead
using our own content file types you are most likely aim at the original engine compatibility. This subject is covered in the very
last section of this manual. %not finished TODO add the said section. Most likely when more features are present.

The actual creation of new files is described in the next chapter. Here we are gonna focus only on details that you need to know
in order to create your first \OCS{} file while full understanding your needs. For now let's jut remember that content files
are created inside the user directory, in the the \textbf{data} subfolder (that is the one special data directory mentioned earlier).

\subparagraph{Dependencies}
Since addon is supposed to change the game it is logical that it also depends on the said game. It simply can not work otherwise.
Just think about it: your modification is changing prize of the iron sword. But what if there is no iron sword in game? That is right:
we get nonsense. What you want to do is to tie your addon to the files you are changing. Those can be either game files (expansion island
for a game) or other addon files (house on the said island). It is a good idea to be dependent only on files that are really changed
in your addon obviously, but sadly there is no other way to achieve this than knowing what you want to do. Again, please remember that
this section of the manual does not cover creating the content files -- it is only theoretical introduction to the subject. For now just
keep in mind that dependencies exist, and is up to you what to decide if your content file should depend on other content file.

Game files are not intend to have any dependencies for a very simple reasons: player is using only one game file (excluding original
and dirty {ESP/ESM} system) at the time and therefore no game file can depend on other game file, and since game file makes the base
for addon files -- it can not depend on addon files.

%\subparagraph{Loading order} %TODO
\paragraph{Project files}
Project files act as containers for data not used by the \OMW{} game engine itself, but still useful for OpenCS. The shining example
of this data category are without doubt record filters (described in the later section of the manual you are reading currently).
As a mod author you probably do not need and/or want to distribute project files at all, they are meant to be used only by you.

As you would imagine, project file makes sense only in combination with actual content files. In fact, each time you start to work
on new content file and project file was not found, it will be created.
Project files extension is, to not surprise ``.project''. The whole name of the project file is the whole name of the content file
with appended extensions. For instance swords.omwaddon file is associated with swords.omwaddon.project file.

%TODO where are they stored.
Project files are stored inside the user directory, in the \textbf{projects} subfolder. This is the path location for both freshly
created project files, and a place where \OCS{} looks for already existing files.

\paragraph{Resources files}
%textures, sounds, whatever
Unless we are talking about the fully text based game, like Zork or Rogue, you are expecting that a video game is using some media files:
models with textures, pictures acting as icons, sounds and everything else. Since content files, no matter if it is ESP, ESM or new \OMW{}
file type do not contain any of those, it is clear that they have to be deliver with a different file. It is also clear that this,
let's call it ``resources file``, have to be supported by the engine. Without code handling those files, it is nothing more than
a mathematical abstraction -- something, that lacks meaning for human beings\footnote{Unless we call programmers a human beings.}.
Therefore this section must cover ways to add resources files to your content file, and point out what is supported. We are going
to do just that. Later, you will learn how to make use of those files in your content.

\subparagraph{Audio}
OpenMW is using {FFmpeg} for audio playback, and so we support every audio type that is supported by this library. This makes a huge list.
Below is only small portion of supported file types.

\begin{description}
 \item mp3 ({MPEG}-1 {Part 3 Layer 3}) popular audio file format and \textit{de facto} standard for storing audio. Used by the \MW{} game.
 \item ogg open source, multimedia container file using high quality vorbis audio codec. Recommended.
\end{description}

\subparagraph{Video}
As in the case of audio files, we are using {FFmepg} to decode video files. The list of supported files is long, we will cover
only the most significant.

\begin{description}
 \item bik videos used by original \MW{} game.
 \item mp4 multimedia container which use more advanced codecs ({MPEG-4 Parts 2,3,10}) with a better audio and video compression rate,
 but also requiring more {CPU} intensive decoding -- this makes it probably less suited for storing sounds in computer games, but good for videos.
 \item webm is a new, shiny and open source video format with excellent compression. It needs quite a lot of processing power to be decoded,
 but since game logic is not running during cut scenes we can recommend it for use with \OMW.
 \item ogv alternative, open source container using theora codec for video and vorbis for audio.
\end{description}

\subparagraph{Textures and images}
Original \MW{} game uses {DDS} and {TGA} files for all kind of two dimensional images and textures alike. In addition, engine supported BMP
files for some reason ({BMP} is a terrible format for a video game). We also support extended set of image files -- including {JPEG} and {PNG}.
JPEG and PNG files can be useful in some cases, for instance JPEG file is a valid option for skybox texture and PNG can useful for masks.
However please, keep in mind that JPEG can grow into large sizes quickly and are not the best option with {DirectX} rendering backend. You probabbly still want 
to use {DDS} files for textures.

%\subparagraph{Meshes} %TODO once we will support something more than just nifs