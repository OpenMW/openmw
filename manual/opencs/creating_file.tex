\section{OpenCS starting dialog}
\subsection{Introduction}
The great day has come. Today, you shall open \OCS{} application. And when you do this, you shall see our starting dialog window that holds three buttons
that can bring both pain and happiness. So just do this, please.

\subsection{Basics}
Back to the manual? Great! As you can see, the starting window holds just three buttons. Since you are already familiar with our files system, they come
to you with no surprise.\\

First, there is a \textbf{Create A New Game} button. Clearly, you should press it when you want to create a game file. Than, what \textbf{Create A New Addon} button do?
Yes! You are right! This button will create any addon content file (and new project file associated with it)! Wonderful! And what the last remaining button do? \textbf{Edit A Content File}? Well, it comes with no surprise that this should be used when you need to alter existing content file, either a game or addon.\\

\paragraph{Selecting Files For New Addon}
As We wrote earlier, both \OMW{} and \OCS{} are operating with dependency idea in mind. As You remember you should only depend on files you are actually using. But how?\\
It is simple. When you click either \textbf{Create new Addon} you will be asked to choose those with a new dialog window. The window is using vertical layout, first you should consider the the top element, the one that allows you to select a game file with drop down menu. Since we are operating on the assumption that there is only one game file loaded at the time, you can depend only on one game file. Next, choose addons that you want to use in your addon with checkboxes.\\

The last thing to do is to name your your addon and click create.

\paragraph{Selecting File for Editing}
Clicking \textbf{Edit A Content File} will show somewhat similar window. Here you should select your Game file with drop down menu. If you want to edit this game file, simply click \textbf{OK} button. If you want to alter addon depending on that file, mark it with checkbox and than click \textbf{Ok} button.

\subsection{Advanced}
If you are paying attention, you noticed any extra icon with wrench. This one will open small settings window. Those are general OpenCS settings. We will cover this is separate section.\\

And that would be it. There is no point spending more time here. We should go forward now.