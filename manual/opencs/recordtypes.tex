\section{Record Types}

\subsection{Introduction}
A gameworld contains many items, such as chests, weapons and monsters. All these items are merely instances of templates that we call \textbf{Objects}. The OpenCS \textbf{Objects} table contains information about each of these template objects, eg. its value and weight in the case of items and an aggression level in the case of NPCs.

Let's go through all Record Types and discuss what you can tell OpenCS about them.

\begin{description}
 \item[Activator:] When the player enters the same cell as this object, a script is started. Often it also has a \textbf{Script} attached to it, though it not mandatory. These scripts are small bits of code written in a special scripting language that OpenCS can read and interpret.
 \item[Potion:] This is a potion that is not self-made. It has an \textbf{Icon} for your inventory, Aside from the self-explanatory \textbf{Weight} and \textbf{Coin Value}, it has an attribute called \textbf{Auto Calc} set to ``False''. This means that the effects of this potion are pre-configured. This does not happen when the player makes their own potion.
 \item[Apparatus:] This is a tool to make potions. Again there's an icon for your inventory as well as a weight and a coin value. It also has a \textbf{Quality} value attached to it: higher the number, the better the effect on your potions will be. The \textbf{Apparatus Type} describes if the item is a Calcinator, Retort, Alembic or Mortar \& Pestle. Each has a different effect on the potion the player makes. For more information on this subject, please refer to the \href{http://www.uesp.net/wiki/Morrowind:Alchemy#Tools}{UESP page on Alchemy Tools}.
 \item[Armor:] This type of item adds \textbf{Enchantment Points} to the mix. Every piece of clothing or armor has a ''pool'' of potential Magicka that gets unlocked when you enchant it. Strong enchantments consume more Magicka from this pool: the stronger the enchantment, the more Enchantment Points each cast will take up. For more information on this subject, please refer to the \href{http://www.uesp.net/wiki/Morrowind:Enchant}{Enchant page on UESP}. \textbf{Health} means the amount of hit points this piece of armor has. If it sustains enough damage, the armor will be destroyed. Finally, \textbf{Armor Value} tells the game how much points to add to the player character's Armor Rating.
 \item[Book:] This includes scrolls and notes. For the game to make the distinction between books and scrolls, an extra property, \textbf{Scroll}, has been added. Under the \textbf{Skill} column a scroll or book can have an in-game skill listed. Reading this item will raise the player's level in that specific skill. For more information on this, please refer to the \href{http://www.uesp.net/wiki/Morrowind:Skill_Books}{Skill Books page on UESP}.
 \item[Clothing:] These items work just like Armors, but confer no protective properties. Rather than ``Armor Type'', these items have a ``Clothing Type''.
 \item[Container:] This is all the stuff that stores items, from chests to sacks to plants. Its \textbf{Capacity} shows how much stuff you can put in the container. You can compare it to the maximum allowed load a player character can carry (who will get over-encumbered and unable to move if he crosses this threshold). A container, however, will just refuse to take the item in question when it gets ''over-encumbered''. \textbf{Organic Container}s are containers such as plants. Containers that \textbf{Respawn} are not safe to store stuff in. After a certain amount of time they will reset to their default contents, meaning that everything in it is gone forever.
 \item[Creature:] These can be monsters, animals and the like. 
 
\end{description}
